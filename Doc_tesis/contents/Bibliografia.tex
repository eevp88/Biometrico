%----------------------------------------------------------
%----------------------------------------------------------
% BIBLIOGRAFÍA

\begin{spacing}{1.0}
\begin{thebibliography}{99}  

\bibitem[Ind09]{Ind09}
Electro Industria. (2009).
\newblock Robótica en Chile. Cada vez más cerca de la automatización total.
\newblock Disponible en \url{http://www.emb.cl/electroindustria/articulo.mvc?xid=1269&tip=9}
\newblock Consultado el 24 de Marzo de 2014.

\bibitem[Osh14]{Osh14}
OSHW-open source hardware. (2014).
\newblock From Definition of Free Cultural Works.
\newblock Disponible en \url{http://freedomdefined.org/OSHW}
\newblock Consultado el 27 de Marzo de 2014.

\bibitem[Del07]{Del07}
Antonio Delgado. (2007).
\newblock ¿Qué es el hardware libre?
\newblock Eroski Consumer
\newblock Disponible en \url{http://www.consumer.es/web/es/tecnologia/hardware/2007/11/20/171514.php}.
\newblock Consultado el 27 de Marzo de 2014.

\bibitem[Car08]{Car08}
Carrasco Livio.  (2008).
\newblock Cancerbero: Prototipo de Control de Acceso utilizando Gestión de espacios mediante Dispositivos Contactless, Smartcard y Biometría.
\newblock Escuela de Ingenieria Civil en Informática
\newblock Universidad Austral  de Chile. 

\bibitem[Vie06]{Vie06}
Vielhauer C. (2006).
\newblock Biometric User Authentication for IT Security: From Fundamentals to Handwriting.
\newblock Springer.

\bibitem[Way00]{Way00}
Wayman J.L. (2000).
\newblock National Biometric Test Center - Collected Works Version 1.2
\newblock Disponible en \url{http://www.engr.sjsu.edu/biometrics/nbtccw.pdf}
\newblock Consultado el 16 de Abril de 2014.


\bibitem[Zha00]{Zha00}
Zhang D. (2000).
\newblock Automated Biometrics, Kluwer
\newblock Springer.


\bibitem[Way99]{Way99}
Wayman J.L. (1999).
\newblock Technical Testing and Evaluation of Biometric Identification Devices.
\newblock Kluwer Academic Publishers.
\newblock Boston, MA, U.S.A, pp. 345 -368

\bibitem[Bal97]{Bal97}
Ballan M, Sakarya F.A. \& Evans B.L. (1997).
\newblock A Fingerprint Classification Technique Using Directional Images Systems and Computers.
\newblock Signals, Systems \& Computers.
\newblock Vol. 1, pp. 101 – 104
\newblock Pacific Grove, CA USA


\bibitem[Olg99]{Olg99}
Patricio Olguín S. (1999).
\newblock Sensores Biometricos.
\newblock Revista electrónica de la escuela de ingeniería eléctrica - Facultad de Ingeniería - Universidad Central Venzuela.
\newblock Disponible \url{http://neutron.ing.ucv.ve/revista-e/No6/Olguin\%20Patricio/SEN_BIOMETRICOS.html}
\newblock Consultado el 3 de Julio de 2014


\bibitem[Rod97]{Rod97}
Roddy A. \& Stosz J., (1997).
\newblock Fingerprint features-statistical analysis and system performance estimates
\newblock Proceedings of the IEEE
\newblock Disponible en \url{http://ieeexplore.ieee.org/stamp/stamp.jsp?tp=&arnumber=628710&isnumber=13673}
\newblock Consultado el 3 de Julio de 2014

\bibitem[Mal03]{Mal03}
Maltoni D., Maio D., Jain A.K. \& Prabhakar S. (2003).
\newblock Handbook of Fingerprint Recognition
\newblock Springer, New York, U.S.A

%------------------------------------------------------------------------------------------



\bibitem[Cis11]{Cis11}
Cisco (2011).
\newblock Big Data in the Enterprise: Network Design Considerations.
\newblock Disponible en \url{http://www.cisco.com/en/US/prod/collateral/switches/ps9441/ps9670/white_paper_c11-690561.pdf}.
\newblock Consultado el 20 de Agosto de 2013.

\bibitem[Com09]{Com09}
Commission of the European Communities (2009).
\newblock Internet of Things - An action plan for Europe.
\newblock Disponible en \url{http://eur-lex.europa.eu/LexUriServ/LexUriServ.do?uri=COM:2009:0278:FIN:EN:PDF}.
\newblock Consultado el 01 de Agosto de 2013.

\bibitem[Dee98]{Dee98}
Deering, S. \& Hinden, R. (1998).
\newblock RFC 2460. Internet Protocol, Version 6 (IPv6) Specification. Internet Engineering Task Force.
\newblock Disponible en \url{http://www.ietf.org/rfc/rfc2460.txt}.
\newblock Consultado el 07 de Agosto de 2013.

\bibitem[Del07]{Del07}
Delclós T. (2007).
\newblock El reto del `Internet de las Cosas'. Diario El País.
\newblock Disponible en \url{http://elpais.com/diario/2007/05/17/ciberpais/1179368665_850215.html}.
\newblock Consultado el 01 de Agosto de 2013.

\bibitem[Dun10]{Dun10}
Dunkels A. \& Vasseur JP. (2010).
\newblock White paper \#1: Why IP. IP for Smart Objects.
\newblock Internet Protocol for Smart Objects (IPSO) Alliance.
\newblock Disponible en \url{http://www.ipso-alliance.org/white-papers}.
\newblock Consultado el 06 de Septiembre de 2013.

\bibitem[Els11]{Els11}
Elster Group (2011).
\newblock A Standardized and Flexible IPv6 Architecture for Field Area Networks.
\newblock Smart Grid Last Mile Infrastructure.
\newblock Disponible en \url{http://www.elster.com/assets/downloads/IP-arch-SG-WP-clean-final-112211.pdf}.
\newblock Consultado el 09 de Septiembre de 2013.

\bibitem[Esh]{Esh}
Echaveguren T., Subiabre M., Echaveguren E., \& León C. (n.d.).
\newblock Proposición de un Subsistema de Información para el Sistema de Gestión de Puentes MAPRA.
\newblock Disponible en \url{http://www2.udec.cl/~provial/trabajos_pdf/11TomasEchavegurenSistemapuentesMapra.pdf}.
\newblock Consultado el 07 de Agosto de 2013.

\bibitem[Eur]{Eur}
European Research Cluster on the Internet of Things (n.d.).
\newblock Disponible en \url{http://www.internet-of-things-research.eu/}.
\newblock Consultado el 07 de Agosto de 2013.

\bibitem[Gar09]{Gar09}
Gracía E. (2009).
\newblock Implementación de Protocolos de Transporte en Redes de Sensores.
\newblock Escuela Técnica Superior de Ingeniería de Telecomunicación de Barcelona,
\newblock Universidad Politécnica de Cataluña.

\bibitem[Gut04]{Gut04}
Gutiérrez J., Barrett R. \& Callaway E. (2004).
\newblock Low-rate Wireless Personal Area Networks: Enabling Wireless Sensors with IEEE 802.15.4.
\newblock Institute of Electrical and Electronics Engineers, New York.

\bibitem[Hin06]{Hin06}
Hinden R. \& Deering S. (2006).
\newblock RFC 4291. IP Version 6 Addressing Architecture. Internet Engineering Task Force.
\newblock Disponible en \url{http://www.ietf.org/rfc/rfc4291.txt}.
\newblock Consultado el 07 de Agosto de 2013.

\bibitem[Hui11]{Hui11}
Hui J. \& Thubert P. (2011).
\newblock RFC 6282. Compression Format for IPv6 Datagrams over IEEE 802.15.4-Based Networks. Internet Engineering Task Force.
\newblock Disponible en \url{http://www.ietf.org/rfc/rfc6282.txt}.
\newblock Consultado el 24 de Septiembre de 2013.

\bibitem[IEE03]{IEE03}
IEEE Standards Association (2003).
\newblock IEEE Standard 802.15.4-2003.
\newblock Disponible en \url{http://standards.ieee.org/getieee802/download/802.15.4-2003.pdf}.
\newblock Consultado el 22 de Agosto de 2013.

\bibitem[IMS12]{IMS12}
IMS Research (2012).
\newblock Internet Connected Devices Approaching 10 Billion, to exceed 28 Billion by 2020.
\newblock Disponible en \url{http://www.imsresearch.com/press-release/Internet_Connected_Devices_Approaching_10_Billion_to_exceed_28_Billion_by_2020}.
\newblock Consultado el 28 de Agosto de 2013.

\bibitem[IPv11]{IPv11}
IPv6 para Chile (2011).
\newblock Fase de Inteligencia de Mercados y Competitiva. Informe de Tendencias N$^{o}$ 6.
\newblock Disponible en \url{http://www.ipv6.cl/system/files/Informe-de-Tendencias-enero-2011.pdf}.
\newblock Consultado el 09 de Septiembre de 2013.

\bibitem[Jim12]{Jim12}
Jiménez A., Jiménez S., Lozada P. \& Jiménez C. (2012).
\newblock Wireless Sensors Network in the Efficient Management of Greenhouse Crops.
\newblock 2012 Ninth International Conference on Information Technology - New Generations.
\newblock Las Vegas, 680 - 685. 

\bibitem[Kas13]{Kas13}
Kaschel H. \& Iturralde D. (2013).
\newblock Análisis de Mejoras en la Agricultura Aplicando WSN: Cultivo de Rosas.
\newblock XIV Congreso Internacional de Telecomunicaciones SENACITEL 2013,
\newblock Valdivia. 

\bibitem[Kuo07]{Kuo07}
Kuorilehto M., Kohvakka M., Suhonen J., Hämäläinen P., Hännikäinen M. \& Hämäläinen TD. (2007).
\newblock Ultra-Low Energy Wireless Sensor Networks in Practice.
\newblock Wiley, Great Britain.

\bibitem[Kus07]{Kus07}
Kushalnagar N., Montenegro G. \& Schumacher C. (2007).
\newblock RFC 4919. IPv6 over Low-Power Wireless Personal Area Networks (6LoWPANs): Overview, Assumptions, Problem Statement, and Goals. Internet Engineering Task Force.
\newblock Disponible en \url{http://www.ietf.org/rfc/rfc4919.txt}.
\newblock Consultado el 07 de Agosto de 2013.

\bibitem[Lar99]{Lar99}
Larman C. (1999).
\newblock UML y Patrones: Introducción Al Análisis y Diseño Orientado a Objetos.
\newblock Prentice-Hall, México.

\bibitem[Lat12]{Lat12}
Latin America and Caribbean Network Information Centre (2012).
\newblock Estado de IPv4 a fin de 2012.
\newblock Disponible en: \url{http://portalipv6.lacnic.net/estado-de-ipv4-a-fin-de-2012-es/}.
\newblock Consultado el 20 de Agosto de 2013.

\bibitem[Max06]{Max06}
MaxStream (2006).
\newblock XBee$^{TM}$/XBee-PRO$^{TM}$ OEM RF Modules. Product Manual v1.xAx - 802.15.4 Protocol.
\newblock MaxStream, Inc., London.

\bibitem[Mol12]{Mol12}
Molina N. (2012).
\newblock Diseño de un Sistema de Gestión de Puentes bajo Enfoque de Priorización de la Inversión.
\newblock Escuela de Ingeniería Civil en Obras Civiles,
\newblock Universidad Austral de Chile, Valdivia.

\bibitem[Mon07]{Mon07}
Montenegro G., Kushalnagar N., Hui J. \& Culler D. (2007).
\newblock RFC 4944. Transmission of IPv6 Packets over IEEE 802.15.4 Networks. Internet Engineering Task Force.
\newblock Disponible en \url{http://www.ietf.org/rfc/rfc4944.txt}.
\newblock Consultado el 07 de Agosto de 2013.

\bibitem[Nar07]{Nar07}
Narten T., Nordmark E., Simpson W. \& Soliman H. (2007).
\newblock RFC 4861. Neighbor Discovery for IP version 6 (IPv6). Internet Engineering Task Force.
\newblock Disponible en \url{http://www.ietf.org/rfc/rfc4861.txt}.
\newblock Consultado el 07 de Agosto de 2013.

\bibitem[Nat10]{Nat10}
National Institute of Standards and Technology (2010).
\newblock NIST Framework and Roadmap for Smart Grid Interoperability Standards, Release 1.0.
\newblock Disponible en \url{http://www.nist.gov/public_affairs/releases/upload/smartgrid_interoperability_final.pdf}.
\newblock Consultado el 09 de Septiembre de 2013.

\bibitem[NXP11]{NXP11}
NXP Laboratories UK  Ltd  (2011). 
\newblock JenNet-IP Network Protocol Stack. Low-Power Wireless IP Networking for the `Internet of Things'.
\newblock Disponible en \url{http://www.jennic.com/files/product_briefs/JenNet-IP-PBv1.2docx.pdf}.
\newblock Consultado el 01 de Agosto de 2013.

\bibitem[Och]{Och}
Ochoa A. (n.d.).
\newblock Métodos científicos.
\newblock Disponible en \url{http://www.monografias.com/trabajos11/metods/metods.shtml}.
\newblock Consultado el 23 de Septiembre de 2013.

\bibitem[Oya10]{Oya10}
Oyarce A. (2010).
\newblock Guía del Usuario XBEE Series 1.
\newblock Ingeniería MCI Ltda., Santiago.

\bibitem[Peñ13]{Peñ13}
Peña C. \& Ralli C. (2013).
\newblock IPv6: El motor de ``La WEB de las Cosas''. Blog Think Big.
\newblock Disponible en \url{http://blogthinkbig.com/ipv6-motor-internet-de-las-cosas-iot/}.
\newblock Consultado el 27 de Agosto de 2013.

\bibitem[Pos80]{Pos80}
Postel J. (1980).
\newblock RFC 768. User Datagram Protocol. Internet Engineering Task Force.
\newblock Disponible en \url{http://www.ietf.org/rfc/rfc768.txt}.
\newblock Consultado el 06 de Septiembre de 2013.

\bibitem[Rev09]{Rev09}
Reventós  L. (2009).
\newblock El `Internet de las Cosas' ahorraría 200000 muertes anuales en las carreteras europeas. Diario El País, Tecnología.
\newblock Disponible en \url{http://tecnologia.elpais.com/tecnologia/2009/05/20/actualidad/1242810061_850215.html}.
\newblock Consultado el 01 de Agosto de 2013.

\bibitem[Rya01]{Rya01}
Ryall M. (2001).
\newblock Bridge Management.
\newblock Elsevier, Great Britain.

\bibitem[Scr13]{Scr13}
Scrum Manager (2013).
\newblock Scrum Manager BoK (SMBoK).
\newblock Disponible en: \url{http://www.scrummanager.net/bok/index.php?title=Main_Page}.
\newblock Consultado el 25 de Septiembre de 2013.

\bibitem[She09]{She09}
Shelby Z. \& Bormann C. (2009).
\newblock 6LoWPAN: The Wireless Embedded Internet.
\newblock Wiley, Great Britain.

\bibitem[She12]{She12}
Shelby Z., Chakrabarti S., Nordmark E. \& Bormann C. (2012).
\newblock RFC 6775. Neighbor Discovery Optimization for IPv6 over Low-Power Wireless Personal Area Networks (6LoWPANs). Internet Engineering Task Force.
\newblock Disponible en \url{http://www.ietf.org/rfc/rfc6775.txt}.
\newblock Consultado el 24 de Septiembre de 2013.

\bibitem[She13]{She13}
Shelby Z., Hartke K. \& Bormann C. (2013).
\newblock draft-ietf-core-coap-18. Constrained Application Protocol (CoAP). Internet Engineering Task Force.
\newblock Disponible en \url{http://tools.ietf.org/id/draft-ietf-core-coap-18.txt}.
\newblock Consultado el 01 de Octubre de 2013.

\bibitem[Taf12]{Taf12}
Taffernaberry  C. (2012).
\newblock 6LoWPAN: IPv6 for Wireless Sensor Network.
\newblock Simposio Argentino de Sistemas Embebidos (SASE).
\newblock Disponible en \url{http://www.sase.com.ar/2012/files/2012/09/4-2012-SASE-6lowpan.pdf}.
\newblock Consultado el 21 de Agosto de 2013. 

\bibitem[Tel13]{Tel13}
Telecom Bretagne (2013).
\newblock Arduino $\mu$IPv6 Stack.
\newblock Disponible en \url{https://github.com/telecombretagne/Arduino-IPv6Stack/wiki}.
\newblock Consultado el 24 de Septiembre de 2013. 

\bibitem[Win12]{Win12}
Winter T., Thubert P., Brandt A., Hui J., Kelsey R., Levis P., Pister K., Struik R., Vasseur JP \& Alexander R. (2012). 
\newblock RFC 6550. RPL: IPv6 Routing Protocol for Low-Power and Lossy Networks . Internet Engineering Task Force.
\newblock Available: \url{http://www.ietf.org/rfc/rfc6550.txt}.
\newblock Consultado el 29 de Agosto de 2013.

\bibitem[Yu06]{Yu06}
Yu Y., Prasanna VK., \& Krishnamachari B. (2006).
\newblock Information Processing and Routing in Wireless Sensor Networks.
\newblock World Scientific Publishing Co. Pte. Ltd., Singapore.

\bibitem[Zia08]{Zia08}
Ziadé T. (2008).
\newblock Expert Python Programming. Best practices for designing, coding, and distributing your Python software.
\newblock Packt Publishing, Birmingham.

\end{thebibliography}	
\end{spacing}

%----------------------------------------------------------
%----------------------------------------------------------
