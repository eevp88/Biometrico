%----------------------------------------------------------
%----------------------------------------------------------
% INTRODUCCIÓN

En la búsqueda de soluciones más eficientes y a bajo costo, conocer de componentes electrónicos como sensores (luminosidad, humedad, temperatura, etc.), microcontroladores, servo motores, pantallas LCD, entre otros, son herramientas poderosas debido a las diferentes problemáticas que pueden dar solución, integrar estas tecnologías dan un gran valor agregado a cualquier tipo proyecto, permitiendo solucionar en el ambiente del usuario diferentes problemáticas de manera útil y efectiva.

Para lograr que este dispositivo interactúe con el ambiente donde está inmerso y se comunique con una base de datos de manera local o de forma remota a través de Internet se hace necesario el desarrollo de piezas de software, capaz de procesar la información recolectada, ya sea en tiempo real o cuando sea necesario, dando de esta manera las funcionalidades necesaria para cubrir las necesidades demandadas.

Para realizar este proyecto se utilizara plataformas de Open Hardware, lo que nos brinda la libertada de construir un dispositivo capaz integrar cualquier tipo de sensor estándar, lo cual nos permite generar el conocimiento para manipular cualquier tipo de sensor.

%--------------------------------------

%--------------------------------------

%--------------------------------------

%--------------------------------------
\subsection{Objetivos}

%-----------------------
\subsubsection{Objetivo general}

Construir un dispositivo capas de cuantificar y controlar el flujo de personas de algún espacio físico y con esto gestionar su mantención de mejor manera. El dispositivo debe ser capaz de integrar distintos sensores de reconocimientos (RFID y lector biométrico), comunicarse con un servidor para poder guardar la información en una base de datos y generar algunos reportes.
\vspace{-0.4cm}
%-----------------------
\subsubsection{Objetivos específicos}


\begin{enumerate}
\item Estudiar las diferentes tecnologías de Open Hardware relevantes para este proyecto, incluyendo sensores (RFID y lector biométrico) y microcontrolador.
\item Definir arquitectura del sistema que permita la escalabilidad e integración con otros sistemas, definiendo estándares de comunicación entre otros.
\item Seleccionar la plataforma de Open Hardware y sensores atingentes al proyecto, diseñar piezas de software que permita comunicar el dispositivo con la base de datos.
\item Modelar e implementar la base de datos que permita manipular la información eficientemente.
\item Entender nuevas tecnologías de programación web para implementar un prototipo de software de administrador de espacios.
\item Realizar pruebas evaluación del funcionamiento del sistema.
\end{enumerate}

\vspace{0.5cm}

%--------------------------------------
\subsection{Motivación}
El desarrollar conocimiento de plataformas de Open Hardware, tanto desde el punto de vista de la electrónica, como de metodologías de desarrollo de software, si bien las metodologías de desarrollo de software tradicionales no son las más adecuadas para este tipo de plataformas, es importante explorar como adaptar está metodologías a este tipo de proyectos.

Una de las áreas donde existe una brecha considerable en la formación como estudiantes, es en la solución de problemas en industrias manufactureras, no desde el punto de vista de la gestión o administración, sino en cubrir funcionalidades prácticas que presenten problemas dentro de los procesos productivos, como por ejemplo detener un motor si la temperatura de él es mayor a 100 grados Celsius o detener una prensa hidráulica si ocurre algún improvisto en la línea de producción, esta área sea transformado en un mercado creciente dentro de este tipo de industria\cite{Ind09} y representa un potencial nicho de mercada para futuros emprendimientos en el área de tecnologías de información y comunicación.

Además poder realizar algunas estadísticas sobre estas máquinas, ayudando a la gestión de estás misma, contribuyendo a los objetivos estratégicos del cliente, con este proyecto en algún grado se acortará esta brecha obteniendo datos del ambiente para luego procesarlos y
automatizar procesos.
\newpage
\subsection{Impacto}

Como este proyecto estará desarrollado en plataformas de Open Hardware y Open Software generará una base de conocimiento con respecto a la integración de dispositivos electrónicos a motores de bases de datos y plataformas web para manejar esta información, ahorrando tiempo y dinero, si en algún momento se requiere de este conocimiento para realizar algún producto de TIC que necesite manejar variables del ambiente en donde se encuentran inmersos y controlar otros dispositivos.

%----------------------------------------------------------
%----------------------------------------------------------
